%______________________________________________________________________________
% main.tex

\input{preamble12-screen.tex}
\hypersetup{%
    pdfauthor={Mike Pierce}%
   ,pdftitle={Math N16B Homework One, Summer 2021}%
   ,pdfkeywords={Pierce,MathN16B,16B,N16B,Calculus,Integration,Berkeley}%
}
\usepackage{fourier}
\input{accessible-colors.tex}
\input{newcommand.tex}
\input{newenvironment.tex}
\pagestyle{empty}


\begin{document}

\begin{center}
    {\Huge{Homework One}}
    \\ \footnotesize{Analytic Geometry and Calculus}
    \\ \footnotesize{UC Berkeley Math N16B, Summer 2021}
\end{center}
\vspace{2em}

Upload your responses to the prompts marked
(\textsc{\textcolor{magenta}{Submit}})
to Gradescope before 8pm Friday; 
you will receive feedback on these.
\begin{center}
    \href{https://www.gradescope.com/courses/275664}%
    {\texttt{gradescope.com/courses/275664}}
\end{center}
The rest of the exercises you should complete at your discretion.
Note that \emph{Calculus with Applications, 11th Edition} 
has some select solutions, usually to odd-numbered exercises, in the back.


\section*{Goals this Week}

Here are some goals you should have in mind while exercising:
\begin{enumerate}
    \item Be able to identify when an integral will need to be
        integrated ``by parts'', and effectively do it.
    \item Be able to visualize (and draw) the three-dimensional solids
        described in this homework, and write down an integral
        that represents their volume. Understand that the integrand
        in these integrals represents a cross-sectional area, 
        and integrating that area along an axis
        calculates how much volume you're accumulating.
    \item Be able to set up integrals that calculate total capital (money)
        accumulated given only information about the money flow
        and any interest rate imposed on that money.
\end{enumerate}

\newpage

\section*{Exercises}

\begin{enumerate}
    %%%%% NOTE THAT THESE PAGE NUMBERS AND ALL THE NUMBERS 
    %%%%% IN THIS ASSIGNMENT CORRESPOND TO THE GLOBAL EDITION OF THE TEXTBOOK.
    \item % Integration by parts
        From Chapter 8.1 of \emph{Calculus with Applications, 11th Edition}
        (page 473) work through the exercises:
        \begin{center}
            1--12 odd
            (\textsc{\textcolor{magenta}{Submit}})
            \quad
            13--22*
            \quad
            33
            \quad
            37 
            (\textsc{\textcolor{magenta}{Submit}})
            \quad
            41
        \end{center}
        *just make sure you know \emph{how you'd start} 
        each of those integrals.

    \item % Volumes of solids of revolution, and average value
        From Chapter 8.2 of \emph{Calculus with Applications, 11th Edition}
        (page 480) work through the exercises:
        \begin{center}
            1--18 every-other-odd
            \quad
            24--37 odd
            \quad
            35
            \quad
            40
        \end{center}

    \item \label{ex:methods}
        There are two methods you could use 
        to find the volume of a solid of revolution:
        the ``disk''/``washer'' method and the ``shell'' method;
        for one you integrate with respect to $x$, 
        and the other with respect to $y$.
        Find the volume of the solid formed by rotating the region bound by 
        the graphs of $y = \ln(x)$, $y=2$, the $x$-axis, and the $y$-axis
        about the $x$-axis using both of these methods. 
        Then find the volume of the solid formed by rotating this same region
        about the $y$-axis using both of these methods.
        See 
        \begin{center}
            \footnotesize
            %\scriptsize
            %\tiny
            \href{https://math.ucr.edu/~mpierce/teaching/biocalc-integral/docs/solid-of-revolution.pdf}%
            {\texttt{math.ucr.edu/\textasciitilde{}mpierce/7b/docs/solid-of-revolution.pdf}}
        \end{center}
        for my write-up of these calculations.

    \item
        (\textsc{\textcolor{magenta}{Submit}})
        Calculate the following general geometric volume formulas.
        Drawing a picture of the solid being described
        will help you set up the correct integral.
        \begin{enumerate}
            \item (\textsc{A Sphere})
                Find the volume of a sphere of radius $1$
                by taking the half-circular region of radius $1$
                that lives below the curve $y=\sqrt{1-x^2}$
                and rotating it about the $x$-axis.
                Then alter this calculation to find
                the general formula for the volume of a sphere of radius $r$.
            \item (\textsc{A Cone})
                Determine a formula for the volume of a right-circular cone
                (a cone with a circular base such that that
                the ``point'' of the cone is directly over the center of the base)
                with height $h$ and circular base of radius $r$.
                How would the volume change in the case
                that the ``point'' is not directly over the center of the base?
            \item (\textsc{A Square-Based Pyramid})
                Determine a formula for the volume of a square-based pyramid
                with height $h$ and with a base of side-length $\ell$.
        \end{enumerate}

    \item 
        (\textsc{Spivak})
        Imagine a solid that has circular base 
        with diameter $\overline{AB}$ with length $\ell$ such that each plane 
        that is perpendicular to $\overline{AB}$ intersects the solid in a square. 
        Express the volume of this solid as an integral 
        and then evaluate the integral.

    \item 
        (\textsc{Challenge})
        You have a bowl full of water, the shape of which 
        is exactly half of a sphere of radius $r$.
        You tilt the bowl thirty degrees, spilling out some of the water.
        What is the volume of the remaining water?

    \item % Money flow
        From Chapter 8.3 of \emph{Calculus with Applications, 11th Edition}
        (page 489) work through the exercises:
        \begin{center}
            15
            \quad
            17
            \quad
            19
        \end{center}

    \item (\textsc{Recreational}) \label{ex:rec}
        Fifty natural numbers are written in such a way 
        so that sum of any four consecutive numbers is 53. 
        The first number is 3, the 19th number is eight times the 13th number, 
        and the 28th number is five times the 37th number. 
        What is the 44th number?
        %\href{here}{https://math.ucr.edu/~mpierce/teaching/recreation/}

\end{enumerate}

\end{document}

